\section{Introdução}

\subsection{Problema}
\begin{frame}{Problema Apresentado}
    \textbf{Problema do artigo:}\\[2mm]
    A irradiância solar é fundamental para dimensionamento e operação de sistemas de energia solar. \\
    Porém, medir irradiância diretamente é caro e tecnicamente complexo, e muitos países - especialmente em desenvolvimento - possuem poucas estações de medição.\\
    \begin{itemize}
        \item Medir irradiância solar diretamente é caro e tecnicamente complexo.
        \item Muitos países em desenvolvimento possuem poucas estações de medição.
        \item Falta de dados consistentes dificulta estimar o potencial solar local.
        \item Não está claro qual modelo de IA funciona melhor para cada tipo de dado e timestep.
    \end{itemize}
\end{frame}

\subsection{Objetivo}
\begin{frame}{Objetivo}
    \textbf{Objetivo do artigo:}\\[2mm]
    Aplicar diferentes modelos de Redes Neuraris Artificiais e técnicas sofisticadas de Aprendizado de Máquina para prever radiação solar em diversas regiões da África.\\ 
    A ideia é auxiliar na aplicação de energia solar em países subdesenvolvidos visto que a medição de radiação solar costuma ser cara e escassa.\\ 
    Países analisados: \textit{Algéria, Nigéria, República Centro-Africana, Senegal, Egito e África do Sul}.\\[4mm]
   

\end{frame}
    
\subsection{Ferramentas do Artigo}
\begin{frame}{Ferramentas de IA e Machine Learning usadas}
    No artigo foram utilizadas algumas ferramentas de IA e machine learning para predição do valor de irradiação solar, essas ferramentas são:
    MLR, PLR, DTR, RFR e XGBoost que representão os algoritmos de Machine Leaning, não são exatamente redes neurais, porém podem ser utilizados juntos.
    ANN, CNN, LSTM e híbridos (CNN-ANN, CNN-LSTM-ANN) que são respectivamente algoritmos de redes neurais que de fato utilizam essas técnicas.

    No nosso modelo acabamos por optar usar ANN ou MLP, por possuir uma implementação um pouco mais fácil
        
    O Data Set utilizafo no artigo foi originado de três bases diferentes: TMY, SARAH e WB-ESMAP. Nos iremos explicar sobre eles mais a frente.
    
    Por fim, temos o sistema de avaliação utilizado que foi RMSE, MAE e correlação.
    
\end{frame}

