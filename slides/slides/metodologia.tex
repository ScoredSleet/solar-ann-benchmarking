\section{Metodologia}

\subsection{Ambiente Computacional e Arquitetura}

\begin{frame}{Arquitetura do Artigo: Artificial Neural Network (ANN)}

    \begin{columns}[t]
        % --- COLUNA 1: Parâmetros de Treinamento ---
        \begin{column}{0.48\textwidth}
            \begin{block}{Parâmetros de Treinamento}
                \begin{itemize}
                    \item \textbf{Framework:} Python TensorFlow \& Keras API.
                    \item \textbf{Otimizador:} \textbf{Adam} (Adaptive Moment Estimation).
                    \item \textbf{Função de Perda:} \textbf{MSE} Média dos erros quadrados dos valores preditos e alvos
                    \item \textbf{Hardware:} Core i7, 2.20 GHz sistema com 16 GB RAM e GTX1060 6 GB.
                \end{itemize}
            \end{block}
        \end{column}

        % --- COLUNA 2: Estrutura da Rede ---
        \begin{column}{0.48\textwidth}
            \begin{block}{Topologia da Rede}
                \begin{itemize}
                    \item \textbf{Ativação (Oculta):} \textbf{ReLU} ($f(x) = \max(0,x)$), escolhida para garantir não-linearidade e convergência rápida.
                    \item \textbf{Ativação (Saída):} \textbf{Sigmoid} ($ f(x) = \frac{1}{1 + \mathrm{e}^{-x}} $) para modelar a distribuição de probabilidade.
                \end{itemize}
            \end{block}
        \end{column}
        
    \end{columns}
\end{frame}

\begin{frame}{Arquitetura do Modelo: Artificial Neural Network (ANN)}
    \begin{columns}[t]
    
        % --- COLUNA 1: Parâmetros de Treinamento ---
        \begin{column}{0.48\textwidth}
            \begin{block}{Parâmetros de Treinamento}
                \begin{itemize}
                    \item \textbf{Framework:} Python PyTorch.
                    \item \textbf{Otimizador:} \textbf{Adam} (Adaptive Moment Estimation).
                    \item \textbf{Função de Perda:} \textbf{MSE} Média dos erros quadrados dos valores preditos e alvos
                    \item \textbf{Hardware:} Google Colab com Intel Xeon @ 2.00GHz ou 2.20GHz, 12.7 GB Ram  e NVIDIA T4/K80 .
                \end{itemize}
            \end{block}
        \end{column}

        % --- COLUNA 2: Estrutura da Rede ---
        \begin{column}{0.48\textwidth}
            \begin{block}{Topologia da Rede}
                \begin{itemize}
                    \item \textbf{Ativação (Oculta):} \textbf{ReLU} ($f(x) = \max(0,x)$), escolhida para garantir não-linearidade e convergência rápida.
                    \item \textbf{Ativação (Saída):} \textbf{Sigmoid} ($ f(x) = \frac{1}{1 + \mathrm{e}^{-x}} $) para modelar a distribuição de probabilidade.
                \end{itemize}
            \end{block}
        \end{column}
        
    \end{columns}
\end{frame}

\subsection{Configuração Experimental e Hiperparâmetros}
% --- SLIDE 1: Primeiros 4 Locais ---
\begin{frame}{Hiperparâmetros dos Modelos - Dataset SARAH}
    \begin{table}[]
        \centering
        \tiny \setlength{\tabcolsep}{4pt}
        \resizebox{\textwidth}{!}{%
            \begin{tabular}{l l c c c c c}
                \toprule
                \textbf{Location} & \textbf{Target} & \textbf{Model} & \textbf{Layers} 
                & \textbf{Neurons (Architecture)} & \textbf{Batch} & \textbf{Epoch} \\ 
                \midrule

                Nigeria - Abuja & DNI & ANN & 3 & [200, 200, 50] & 128 & 100 \\
                \midrule

                Nigeria - Akure & DNI & ANN & 3 & [200, 200, 100] & 128 & 100 \\
                %\midrule

                %Algeria - Tamarasset & GSR & ANN & 3 & [100, 100, 50] & 512 & 100 \\
                %\midrule

                %Nigeria - Borno & DSR & ANN & 2 & [100, 50] & 512 & 50 \\
                \bottomrule
            \end{tabular}%
        }
    \end{table}
\end{frame}


% --- SLIDE 2: Outros Países ---
\begin{frame}{Hiperparâmetros dos Modelos - Datset TMY}
    \begin{table}[]
        \centering
        \tiny \setlength{\tabcolsep}{4pt}
        \resizebox{\textwidth}{!}{%
            \begin{tabular}{l l c c c c c}
                \toprule
                \textbf{Location} & \textbf{Target} & \textbf{Model} 
                & \textbf{Layers} & \textbf{Neurons (Architecture)}
                & \textbf{Batch} & \textbf{Epoch} \\ 
                \midrule

                Central African Republic & GSR & ANN 
                & 3 & [200, 200, 100] & 512 & 30 \\
                \midrule

                Egypt & GSR & ANN
                & 3 & [200, 200, 100] & 128 & 7 \\
                \midrule

                South Africa & GSR & ANN
                & 2 & [100, 50] & 512 & 20 \\ \midrule

                Algeria  & GSR & ANN & 3 & [100, 100, 50] & 512 & 100 \\
                \midrule

                Nigeria  & DSR & ANN & 2 & [100, 50] & 512 & 50 \\
                \bottomrule
            \end{tabular}%
        }
    \end{table}
\end{frame}


% --- SLIDE 3: Senegal (Touba) ---
\begin{frame}{Hiperparâmetros dos Modelos — WB-ESMAP}
    \begin{table}[]
        \centering
        \tiny \setlength{\tabcolsep}{4pt}
        \resizebox{\textwidth}{!}{%
            \begin{tabular}{l l l c c c c}
                \toprule
                \textbf{Location} & \textbf{Target} & \textbf{Model}
                & \textbf{Layers} & \textbf{Neurons (Architecture)}
                & \textbf{Batch} & \textbf{Epoch} \\
                \midrule

                % --- TOUBA ---
                Touba & DHI\textsubscript{RSI} & ANN
                & 2 & [50(0.25), 50(0.25)] & 128 & 20 \\

                Touba & GHI\textsubscript{pyr} & ANN
                & 2 & [100(0.25), 200(0.25)] & 128 & 40 \\

                Touba & GHI\textsubscript{Sil} & ANN
                & 2 & [100, 50] & 128 & 30 \\
                \midrule

                % --- FATICK ---
                Fatick & DHI\textsubscript{RSI} & ANN
                & 2 & [100(0.25), 50(0.25)] & 128 & 20 \\

                Fatick & GHI\textsubscript{pyr} & ANN
                & 2 & [100, 50(0.25)] & 128 & 50 \\

                Fatick & GHI\textsubscript{Sil} & ANN
                & 2 & [50(0.25), 50(0.25)] & 128 & 150 \\

                \bottomrule
            \end{tabular}%
        }
    \end{table}
\end{frame}


\subsection{Explicação dos Dados e Fontes de Dados}
\begin{frame}{Sobre os Datasets Utilizados - TMY}

     \textbf{TMY (Typical Meteorological Year):}
    \begin{itemize}
        \item Variáveis: ano, mês, dia, hora, elevação solar, temperatura do ar, velocidade do vento a 10 m
        \item Tamanho aproximado: \textbf{105 mil linhas} \\
        \item \textbf{Cidades}: Northern Cape, Vakaga, Mut, Tamanrasset e Borno
        \item \textbf{GSR}: Radiação solar global (energia direta do Sol atingindo o solo)
        \item \textbf{DSR}: Radiação solar difusa (energia indireta espalhada por nuvens/atmosfera)
    \end{itemize}

    %\textbf{Tipos de irradiância previstos por cidade:}\\[2mm]
    
    %\textbf{Northern Cape, Vakaga, Mut, Tamanrasset e Borno:}
    %\begin{itemize}
        %\item \textbf{GSR}: Radiação solar global (energia direta do Sol atingindo o solo)
        %\item \textbf{DSR}: Radiação solar difusa (energia indireta espalhada por nuvens/atmosfera)
    %\end{itemize}
    
    %\textbf{ Touba e Fatick (Senegal):}
    %\begin{itemize}
        %\item \textbf{DHI\textsubscript{RSI}}: Radiação difusa horizontal medida por \textit{Rotating Shadowband Irradiometer}
        %\item \textbf{GHI\textsubscript{Sil}}: Radiação global (direta + difusa) medida com sensor de silício
        %\item \textbf{GHI\textsubscript{Pyr}}: Radiação global medida com piranômetro de termopilha (alta precisão e caro, o estado-da-arte)
    %\end{itemize}

     %\textbf{ Akure e Abuja (Nigeria):}
    %\begin{itemize}
        %\item \textbf{DNI}: Fluxo média diário de radiação na superfície normal em direção ao Sol.
        
    %\end{itemize}
\end{frame}


\begin{frame}{Sobre os Datasets Utilizados - WB-ESMAP}
    
    \textbf{WB-ESMAP (World Bank - ESMAP):}
    \begin{itemize}
        \item Variáveis: ano, mês, dia, hora, minuto, temperatura do ar, umidade relativa, velocidade e direção do vento, velocidade calculada do vento, precipitação, pressão barométrica, limpeza do sensor, precipitação e pressão barométrica.
        \item Tamanho aproximado: \textbf{566 mil linhas}
        \item \textbf{Cidades}: Touba e Fatick (Senegal)
        \item \textbf{DHI\textsubscript{RSI}}: Radiação difusa horizontal medida por \textit{Rotating Shadowband Irradiometer}
        \item \textbf{GHI\textsubscript{Sil}}: Radiação global (direta + difusa) medida com sensor de silício
        \item \textbf{GHI\textsubscript{Pyr}}: Radiação global medida com piranômetro de termopilha (alta precisão e caro, o estado-da-arte)
    \end{itemize}
    

    
\end{frame}


\begin{frame}{Sobre os Datasets Utilizados - WB-ESMAP}
        \textbf{SARAH (Surface Radiation Data Set - Heliosat):}
    \begin{itemize}
        \item Variáveis: ano, mês, dia e duração de luz do Sol
        \item Tamanho aproximado: \textbf{12 mil linhas}
        \item \textbf{Cidades}: Akure e Abuja (Nigeria)
        \item \textbf{DNI}: Fluxo média diário de radiação na superfície normal em direção ao Sol.
    \end{itemize}
\end{frame}


\subsection{Métricas de Avaliação de Desempenho}
% --- SLIDE 1: Métricas de Erro (MSE e RMSE) ---
\begin{frame}{Métricas de Avaliação: Erros}
    \small
    
    O autor avaliou a precisão do modelo utilizando métricas de erro para quantificar a divergência entre os dados:

    \vspace{0.3cm}

    \begin{block}{Média dos Erros Absoluto (MAE)}
        MAE é uma métrica comum usada para mensurar o valor absoluto entre o valor predito e o valor alvo, uma falha dele é não proporcionar uma visualização tão próxima dos dados. 
        $$ MAE = \frac{1}{N} \sum_{i=1}^{N} |G_i^m - G_i^p| $$
    \end{block}

    \vspace{0.3cm}

    \begin{block}{Raiz da Média dos Erros Quadrados (RMSE)}
        RMSE é a raiz quadrada do MSE, ele trás as métricas de volta as mesmas unidades de medidas dos valores alvos.
        $$ RMSE = \sqrt{\frac{1}{N} \sum_{i=1}^{N} (G_i^m - G_i^p)^2} $$
    \end{block}
\end{frame}

% --- SLIDE 2: Correlação e Legenda ---
\begin{frame}{Métricas de Avaliação: Correlação}
    \textbf{Coeficiente de Relação (r):} \\
    Avalia a força da correlação linear entre os dados medidos e os preditos.
    
    % Fórmula em destaque
    \begin{block}{}
        \footnotesize % Reduz um pouco a fonte SÓ da fórmula se precisar
        $$ 
        r = \left( \frac{ \sum_{i=1}^{N} \left( (G_i^m - \langle G_i^m \rangle) (G_i^p - \langle G_i^p \rangle) \right) }{ \sqrt{\sum_{i=1}^{N} (G_i^m - \langle G_i^m \rangle)^2} \sqrt{\sum_{i=1}^{N} (G_i^p - \langle G_i^p \rangle)^2} } \right)^2 
        $$
    \end{block}

    \vspace{0.5cm}
    \hrule % Linha horizontal para separar
    \vspace{0.2cm}

    % Legenda organizada em colunas para economizar espaço
    \textbf{Legenda dos Termos:}
    \footnotesize % Fonte menor para a legenda
    \begin{columns}[t]
        \begin{column}{0.45\textwidth}
            \begin{itemize}
                \item $N$: Número total de observações.
                \item $G_i^m$: Valor \textbf{medido} na amostra $i$.
                \item $G_i^p$: Valor \textbf{predito} (simulado) na amostra $i$.
            \end{itemize}
        \end{column}
        
        \begin{column}{0.45\textwidth}
            \begin{itemize}
                \item $\langle G^m \rangle$: Média dos valores medidos.
                \item $\langle G^p \rangle$: Média dos valores preditos.
            \end{itemize}
        \end{column}
    \end{columns}
\end{frame}

\begin{frame}{Variações de combinações de parâmetros}

\begin{minipage}{0.55\textwidth}
    \textbf{Utilizamos 2 otimizadores e 3 valores de learning rate:}

    \begin{itemize}
        \item \textbf{ADAM}: momentum adaptativo e ajuste automático de taxa de aprendizado.
        \item \textbf{SGD}: mais simples e com melhor generalização
              quando usado com momentum e weight decay.
        \item \textbf{LRs}: [1e-3, 5e-4, 1e-4].
        \item \textbf{Normalização}: Obtivemos resultados mais consistentes com normalização do dataset.
    \end{itemize}
\end{minipage}
\hfill
\begin{minipage}{0.40\textwidth}
    \centering
    \includegraphics[width=\textwidth]{imagens/train.png} 
\end{minipage}

\end{frame}