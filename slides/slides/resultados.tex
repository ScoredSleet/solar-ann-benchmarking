\section{Resultados}



% - Tabelas do modelo
\newcommand{\TabelaDadosModelo}{
    \begin{tabular}{llccc}
        \toprule
        \textbf{Cidade} & \textbf{Target} & \textbf{MAE} & \textbf{RMSE} & \textbf{R} \\ 
        \midrule
        \multirow{3}{*}{Touba} 
          & dhi\_rsi & 77,51 & 118,67 & 0,58 \\ 
          & ghi\_pyr & 106,79 & 177,72 & 0,85 \\ 
          & ghi\_sil & 74,93 & 116,16 & 0,88 \\ 
        \midrule
        \multirow{3}{*}{Fatick} 
          & dhi\_rsi & 68,32 & 112,16 & 0,62 \\ 
          & ghi\_pyr & 65,26 & 114,51 & 0,88 \\ 
          & ghi\_sil & 77,87 & 128,63 & 0,86 \\ 
        \midrule
        SA Northern Cape    & GSR & 58,81 & 104,26 & 0,89 \\ 
        CAR Vakaga          & GSR & 69,52 & 131,87 & 0,78 \\ 
        Egypt Mut           & GSR & 90,37 & 130,61 & 0,86 \\ 
        Algeria Tamanrasset & GSR & 52,94 & 105,69 & 0,88 \\ 
        Nigeria Borno       & DSR & 21,64 & 36,64 & 0,89 \\
        \midrule
        Abuja               & DNI &  38.88 & 54.75 & 0.74 \\
        Akure               & DNI &  32.49 & 46.94 & 0.68 \\
        \bottomrule
    \end{tabular}
}

% - Tabela do Arigo
\newcommand{\TabelaDadosArtigo}{
    \begin{tabular}{llccc}
        \toprule
        \textbf{Cidade} & \textbf{Target} & \textbf{MAE} & \textbf{RMSE} & \textbf{R} \\ 
        \midrule
        \multirow{3}{*}{Touba} 
          & dhi\_rsi &  75.13 & 105.44 & 0.78 \\ 
          & ghi\_pyr & 124.74 & 191.48 & 0.82 \\ 
          & ghi\_sil & 124.50 & 188.48 & 0.80 \\ 
        \midrule
        \multirow{3}{*}{Fatick} 
          & dhi\_rsi & 86.71 & 124.11 & 0.70 \\ 
          & ghi\_pyr & 144.44 & 206.88 & 0.79 \\ 
          & ghi\_sil & 147.88 & 203.41 & 0.78 \\ 
        \midrule
        SA Northern Cape    & GSR &  34.67 & 93.50 & 0.97 \\ 
        CAR Vakaga          & GSR &  45.56 & 95.94 & 0.96 \\ 
        Egypt Mut           & GSR &  26.13 & 63.62 & 0.99 \\ 
        Algeria Tamanrasset & GSR & 27.59 & 81.96 & 0.98 \\ 
        Nigeria Borno       & DSR &  19.44 & 49.34 & 0.90 \\ 
        \midrule
        Abuja               & DNI &  42.70 & 55.93 & 0.78 \\
        Akure               & DNI &  19.11 & 25.15 & 0.95 \\
        \bottomrule
    \end{tabular}
}


\subsection{Análise Comparativa Numérica}
% --- SLIDE 3: Métricas do modelo ---
%\begin{frame}{Métricas por Cidade (Modelo)}
%    \begin{table}[]
%        \centering
%        % O resizebox ajusta a tabela para caber na largura do slide
%        \resizebox{\textwidth}{!}{\TabelaDadosModelo}
%    \end{table}
%\end{frame}
%
% --- SLIDE 4: Métricas do artigo ---
%\begin{frame}{Métricas por Cidade (Artigo)}
%   \begin{table}[]
%        \centering
%        % O resizebox ajusta a tabela para caber na largura do slide
%        \resizebox{\textwidth}{!}{{\TabelaDadosArtigo}}
%    \end{table}
%\end{frame}


% --- SLIDE 5: Comparação de Métricas de Cidades do Modelo com o Artigo ---

\begin{frame}{Comparação de Métricas por Cidade}

    \begin{block}{}
        A configuração imediata que tivemos foi: ADAM, sem normalização nem ativação sigmoide e lr=1e-3. A primeira execução resultou em:
    \end{block}
    \vspace{0.2cm}
    
    % --- Lado Esquerdo ---
    \begin{minipage}{0.48\textwidth}
        \centering
        \textbf{Resultados (Modelo)}
        \vspace{0.2cm}
        
        \resizebox{\textwidth}{!}{ \TabelaDadosModelo } 
    \end{minipage}
    \hfill
    % --- Lado Direito ---
    \begin{minipage}{0.48\textwidth}
        \centering
        \textbf{Resultados (Artigo)}
        \vspace{0.2cm}
        
        \resizebox{\textwidth}{!}{ \TabelaDadosArtigo }
    \end{minipage}

\end{frame}

\begin{frame}{Comparação de Métricas por Cidade}

    \begin{block}{}
        Já a nossa melhor execução, dentre todas as testadas foi:
        (ATUALIZAR ESSA TABELA)
    \end{block}
    \vspace{0.2cm}
    
    % --- Lado Esquerdo ---
    \begin{minipage}{0.48\textwidth}
        \centering
        \textbf{Resultados (Modelo)}
        \vspace{0.2cm}
        
        \resizebox{\textwidth}{!}{ \TabelaDadosModelo } 
    \end{minipage}
    \hfill
    % --- Lado Direito ---
    \begin{minipage}{0.48\textwidth}
        \centering
        \textbf{Resultados (Artigo)}
        \vspace{0.2cm}
        
        \resizebox{\textwidth}{!}{ \TabelaDadosArtigo }
    \end{minipage}

\end{frame}


% ------- Comparação entre graficos Tamanrasset Exemplo TMY -------
\subsection{Comparação entre gráficos}

\begin{frame}{Comparação de Métricas por Dataset - TMY}

    % ----- Texto no topo -----
        Comparação entre o gráfico apresentado no artigo e o gráfico gerado pela nossa execução. Aqui: Predição de GSR na cidade de Tamanrasset. \\
        >SGD, normalização e  lr=1e-4
    % ----- Imagem do Artigo (em cima) -----
    \begin{center}
        \textbf{Gráfico do Artigo}\\[0.01cm]
        \includegraphics[width=0.57\textwidth]{imagens/grafico_tamanrasset_artigo.png}
    \end{center}
    % ----- Nossa Imagem (embaixo) -----
    \begin{center}
        \textbf{Nosso Gráfico}\\[0.05cm]
        \includegraphics[width=0.57\textwidth]{Algeria Tamanrasset_GSR_sgd_0.0001.png}
    \end{center}

\end{frame}


% ------- Comparação entre graficos Abuja  Exemplo do SARAH ------
\begin{frame}{Comparação de Métricas por Dataset - SARAH}

    % ----- Texto no topo -----
    Comparando o gráfico do artigo com nossos resultados. Cidade de Abuja na Nigéria, prevendo DNI. \\
    >ADAM, normalização e  lr=1e-4
    % ----- Imagem do Artigo (em cima) -----
    \begin{center}
        \textbf{Gráfico do Artigo}\\[0.01cm]
        \includegraphics[width=0.61\textwidth]{grafico_artigo_nigeria_abuja_daily.png}
    \end{center}
    %\vspace{0.1cm}
    % ----- Nossa Imagem (embaixo) -----
    \begin{center}
        \textbf{Nosso Gráfico}\\[0.05cm]
        \includegraphics[width=0.61\textwidth]{Nigeria Abuja_DNI_adam_0.0001.png}
    \end{center}

\end{frame}

% ------- Comparação entre graficos Abuja  Exemplo do WB-ESMAP ------
\begin{frame}{Comparação de Métricas por Dataset - WB-ESMAP}

    % ----- Texto no topo -----
        Comparando o gráfico do artigo com nossos resultados. Cidade de Touba em Senegal, prevendo GHI pyr \\
        >ADAM, normalização e  lr=1e-3
    % ----- Imagem do Artigo (em cima) -----
    \begin{center}
        \textbf{Gráfico do Artigo}\\[0.01cm]
        \includegraphics[width=0.55\textwidth]{grafico_touba_ghi_pyr.png}
    \end{center}
    % ----- Nossa Imagem (embaixo) -----
    \begin{center}
        \textbf{Nosso Gráfico}\\[0.05cm]
        \includegraphics[width=0.55\textwidth]{imagens/Touba_ghi_pyr_adam_0.001.png}
    \end{center}

\end{frame}

% ----- Comparação sigmoide e sem ----------
\begin{frame}{Comparação dos Modelos — Saída Linear vs Sigmoide}

    % ----- Texto no topo -----
    %\begin{block}{}
        Os modelos com sigmoide geram curvas “achatadas” e incapazes de seguir a variação verdadeira. Isso, em todas as execuções.\\
        > Previsão de GHI pyt em Fatick, uso de SGD, normlaização e lr=5e-4
    %\end{block}
    % ----- Imagem do Artigo (em cima) -----
    \begin{center}
        \textbf{Redesem ativação Sigmoide}\\[0.01cm]
        \includegraphics[width=0.61\textwidth]{imagens/Fatick_ghi_pyr_sgd_0.0005_linear.png}
    \end{center}
    % ----- Nossa Imagem (embaixo) -----
    \begin{center}
        \textbf{Rede com ativação Sigmoide}\\[0.05cm]
        \includegraphics[width=0.61\textwidth]{Fatick_ghi_pyr_sgd_0.0005_sigmoide.png}
    \end{center}

\end{frame}

% ----- usou ou não de normalização
\begin{frame}{Uso ou não de normalização}
\begin{itemize}
        \item Tamanho aproximado: \textbf{12 mil linhas}
        \item \textbf{Cidades}: Akure e Abuja (Nigeria)
    \end{itemize}
\end{frame}


% --- Função para chamar a imagem diretamente
\newcommand{\CardImagem}[2]{
    \centering
    % A imagem ocupa a largura disponível (do minipage ou do slide)
    \includegraphics[width=\textwidth]{#1} 
    
    \vspace{0.1cm}
    
    % A legenda em itálico e pequena
    \footnotesize{\textit{#2}}
}

% --- SLIDE 6: Gráficos de comparação ---
\subsection{Análise Visual e Gráfica}
\begin{frame}{Comparação Visual dos Resultados}

    
    % --- Lado Esquerdo (Seu Modelo) ---
    \begin{minipage}[t]{0.48\textwidth}
        \centering
        \textbf{Resultados (Nosso Modelo)}
        \vspace{0.2cm} % Um pequeno espaço entre o título e a imagem
        
        % Chama o comando da sua imagem
        \CardImagem{test.jpg}{Teste}
        
    \end{minipage}
    \hfill % Espaço flexível no meio
    % --- Lado Direito (Artigo) ---
    \begin{minipage}[t]{0.48\textwidth}
        \centering
        \textbf{Resultados (Artigo Base)}
        \vspace{0.2cm} % Um pequeno espaço entre o título e a imagem
        
        % Chama o comando da imagem do artigo
    \end{minipage}

\end{frame}

